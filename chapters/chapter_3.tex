%\section{Literature Review}
\section{Literature Review} 
Glaucoma is one of the most common causes of permanent blindness around the world, from the article [10]. As when the pressure inside the eye is too high in a particular nerve that moment glaucoma will develop and it will also create eye ache. The working mechanisms of the different diagnosis tools like tonometers, gonioscopy, scanning laser tomography, etc are available for the treatment and detection but there are some advantages and disadvantages which sometimes create boundaries. For this, there should be an evaluation of how this works. But with using deep learning the boundaries can be removed. As the XAI concept can be understood by humans which will be closer to the human brain to understand.

\vspace{5mm}
\noindent Recent breakthroughs in machine learning (ML) have the potential to significantly enhance retinal disease screening and diagnostic accuracy according to [11]. The recent most demanding field of explainable artificial intelligence (XAI) attempts to focus on Glaucoma disease. As a result, the necessity for expert-level review in assessing their efficacy becomes unavoidable. In a series of tests, we illustrate the efficacy of our method. XAI is vastly classified in these categories- Application-Grounded Evaluation, Human-Grounded Evaluation, and Functionally Grounded Evaluation. Some of the findings from this paper are as follows: (i) Recent breakthroughs in machine learning promise to significantly enhance retinal disease screening and diagnostic accuracy. Multiple eye illnesses, including diabetic retinopathy, age-related macular degeneration (AMD), glaucoma, and other anomalies associated with retinal diseases, to keep track of their progress. This has been diagnosed with expert-level accuracy using systems built using these methodologies. (ii) XAI's purpose is to decode the decision of Artificial Intelligence which means Deep Learning or Machine learning black box to the extent that it is human-interpretable. (iii) Used two of the most current visual explanation methods to assess the visual explanation on the provided dataset which are SIDU and GRAD-CAM. As a result, in addition to enhancing the tool's accuracy, the concept of trust, as well as the requirement for openness and robustness, emphasizes the need of investigating the impact of expert review in the context of XAI approaches.

\vspace{5mm}
\noindent Computer-aided diagnostics(CADx) tools are still struggling to detect glaucoma eye illness according to author Qaisar Abbas [12]. Glaucoma is the main reason for visual disability in the whole world. His writings revealed that the Softmax linear classifier makes the ultimate judgment to distinguish between glaucoma and non-glaucoma retinal fundus images. Glaucoma-Deep, the suggested method, was evaluated on 1200 retinal pictures gathered from publicly and privately available datasets. Then the sensitivity (SE), specificity (SP), accuracy (ACC), and precision (PRC) statistical measurements were used to evaluate the performance of the Glaucoma-Deep system. In general, the SE of 84.50\%, SP of 98.01\%, ACC of 99\%, and PRC of 84\% values were achieved through this. When compared to other systems, the Nodular-Deep approach produced much better outcomes. As a result, the Glaucoma-Deep system can quickly identify glaucoma eye illness, solving the problem of clinical specialists during large-scale eye-screening processes.

\vspace{5mm}
\noindent Glaucoma is the leading cause of blindness in the world, and there is no treatment [13]. If it is not diagnosed at an early stage, it can surely lead to irreversible blindness. If vision loss is detected early enough, there are treatments available to prevent it. Because it is a significant chronic eye condition that leads to irreversible blindness. Glaucoma has been on the rise in recent years. Faults in the nerve fiber layer of the retina are diagnosed before apparent abnormalities at the head of the optic nerve and defects in the visual field when 40 percent of axons have been irreversibly destroyed. According to the World Health Organization (WHO) and the World Association of Glaucomatologists (WGO), 66.8 million individuals worldwide suffered from glaucoma in 2010, with 6.7 million becoming blind as a result of the disease.

\vspace{5mm}
\noindent Another version of the deep-learning (DL) algorithm was developed in [6] to detect glaucoma disease through extracting several parameters such as 52 total deviation, mean deviation, and pattern standard deviation values. Here the writer used a Deep learning classifier such as a deep feed-forward neural network (FNN). The authors, on the other hand, integrated their DL classifier with older machine learning classifiers including random forests (RF), gradient boosting, support vector machine, and neural network (NN). As a result, the authors provided a deep ensemble solution for glaucoma illness detection. A deep FNN classifier was used to get 92.5 percent of the AUC value, according to the authors.

\vspace{5mm}
\noindent We learned The impact of artificial intelligence in the diagnosis and management of glaucoma from [15]. Computerized automated visual field testing represents a significant improvement in mapping the island of vision, allowing visual field testing to become a cornerstone in diagnosing and managing glaucoma. Goldbaum developed a two-layer neural network for analyzing visual fields in 1994 et al.[8] . This network classified normal and glaucomatous eyes with the same sensitivity (65\% ) and specificity (72\% ) as two glaucoma specialists.

\vspace{5mm}
\noindent According to the writer [16] To diagnose illnesses, different healthcare systems employ content-based picture analysis and computer vision algorithms. Fundus pictures recorded with a fundus camera are used to identify abnormalities in the human eye. Glaucoma is the second most common cause of neurodegenerative sickness among eye illnesses. Glaucoma has no symptoms in its early stages, and if the condition is not treated, it can result in total blindness. Glaucoma can be detected early enough to prevent irreversible visual loss. Although manual inspection of the human eye is a viable option, it is reliant on human effort. The goal of this review article is to provide a complete overview of the numerous varieties of glaucoma, their causes, prospective treatments, publicly accessible image benchmarks, performance measures, and different methodologies based on digital image processing, computer vision, and deep learning. The review paper examines a variety of published research models for detecting glaucoma, ranging from low-level feature extraction to contemporary deep learning developments. The advantages and disadvantages of each strategy are examined in-depth, and the findings of each category are summarized using tabular representations.

\vspace{5mm}
\noindent As previous data shows how glaucoma disease gradually leads to blindness[8]. If we can detect glaucoma early it can be preventable against developing more serious conditions, they claimed. Cup-to-disc Ratio or CDR is an essential clinical indicator for glaucoma diagnosis in their research. Their objective is to develop a system that can provide a proper path to compute CDR results with the highest possible accuracy. They used 44 retinal images from Mettapracharak hospital to evaluate the performance, of which 29 retinal images were patients with no glaucoma and 15 were with glaucoma disease. Which shows impressive accuracy. Then compare the value of CDR which is more than 0.65 is used to access a patient as a possible glaucoma case. CDR value between clinical result and edge detection with power raw transformation approach. Where the proposed method was 5.14\%. The percentage error by using their proposed method for optic disc segmentation, optic cup segmentation and CDR are 2.49\%, 5.8\%, and 5.14\%, respectively. The CDR is a crucial clinical sign for determining a person's risk of developing glaucoma. For this, in their paper, they presented a method to calculate the CDR automatically from fundus images the author added.

\vspace{5mm}
\noindent Following the enormous success of one class of mathematical models, the artificial neural network, artificial intelligence, or AI has risen. Deep learning, a recently invented method, has taken over current scientific discourse, penetrating areas such as physics, chemistry, engineering, biology, and medicine [17]. This leads to a discussion on current solutions and state-of-the-art, with some drawbacks that may limit clinical adoption. Glaucoma is usually linked with increased intraocular pressure (IOP), which affects the overall visual field of the eye over time[18][19]. Research on glaucoma suggests that the disease's development is influenced by several interconnected bodily mechanisms. There are two kinds of glaucoma: open-angle and closed-angle. The angle refers to the length of contact between the iris and the cornea; if the length is long, the related illness is called open-angle glaucoma; if the length is short, it is called closed-angle glaucoma, they added [19]. Not only can glaucoma affect the patient's eyesight, but it's also linked to a hearing disability (Greco et al. in The American Journal of Medicine, 2016) [19]. They reviewed the complete techniques of glaucoma detection using  Deep neural networks.

\vspace{5mm}
\noindent The pathogenesis of glaucoma appears to be dependent on several interconnected pathogenetic mechanisms, including mechanical effects characterized by excessive intraocular pressure, reduced neutrophil produce, hypoxia, excitotoxicity, oxidative stress, and the involvement of autoimmune processes, according to new evidence [19]. Hearing loss has also been linked to the development of glaucoma. In normal-tension glaucoma patients with hearing loss, antiphosphatidylserine antibodies of the immunoglobulin G class were shown to be more prevalent than in normal-tension glaucoma patients with normacusis. The World Health Organization reports that glaucoma affects approximately 60 million people worldwide. By the year 2020, it is expected that approximately 80 million people will suffer from glaucoma, which is anticipated to result in 11.2 million cases of bilateral blindness [20]. This is why it needs to be treated as early as possible according to the authors.

\vspace{5mm}
\noindent The visual fields from an automated perimeter were taught to be interpreted by neural networks. The scientists tested the trained neural networks' capacity to distinguish between normal and glaucoma-affected eyes [21]. After research, we got that, Glaucoma specialists and a trained two-layered network both got around 67 percent of the answers right. The two glaucoma experts had a sensitivity of 59 percent, while the two-layered network had a sensitivity of 65 percent. For the specialists and the two-layered network, the corresponding specificities were 74\% and 71\%, respectively. About 74 percent of the time, the experts and the network agreed, indicating that there was no substantial discrepancy between the testing methodologies. The most relevant visual field locations were discovered using feature analysis and a one-layered network. Here the authors conclude that a neural network may be taught to evaluate visual fields for glaucoma as well as a professional reader. The researchers compared the backpropagation learning approach used by automated neural networks to the methods employed by two glaucoma specialists to identify the center’s 24 degrees automated perimetry visual fields from 60 normal and 60 glaucomatous eyes. However The neural network like deep learning has many limitations like a vast number of pictures must be incorporated in DL algorithms for them to predict with high sensitivity and specificity, moreover obtaining and storing a large number of photos comes with time limits and technological challenges. Furthermore, for such databases to remain current and prevent system-wide algorithm failure, they may need to be updated regularly. And most importantly Because the mechanism of DL's prediction is unclear, it is referred to as a "black box", which clearly shows its limitations. To resolve these issues using the XAI can be a big step towards Glaucoma detection.

\nomenclature{$OCT$}{Optical Coherence Tomography}
\nomenclature{$AMD$}{Age-related Macular Degeneration}
\nomenclature{$CADx$}{Computer-aided diagnostics}