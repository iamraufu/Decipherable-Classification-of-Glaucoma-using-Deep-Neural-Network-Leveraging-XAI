%\section{Conclusion}
\section{Conclusion} 
In this research, to attain the ultimate objective of our study in Glaucoma Diagnosis, the Black
Box model was defined using Explainable Artificial Intelligence (XAI). As we have introduced
our work we have done so far in this research paper. Through this research, we were led to more
glaucoma diagnosis acceptability. Notably, glaucoma can take away the vision of a patient which
is why it’s more important to work for a better solution for early stage detection of glaucoma
disease. For this reason, using the XAI, recognition, and treatment of Glaucoma can bring an
immense change in the system which is very important, as reducing the number of blindness
caused by glaucoma through early proper detection and treatment of the disease is going to be a
huge success to celebrate. Because around the world 1 out of 15 people are blind because of it.
Statistics show that even with the treatment 15% to 20% of the patients become blind. To serve
our purpose we have used VGG-16, VGG-19, DenseNet121, InceptionV3 and ResNet50 models
for our study. Every model was compiled with Adam optimizer with the learning rate of 1e-5 in
50 epochs. If we look at the score which is validation accuracy for our models we can see that in
InceptionV3 we got 86.4\% accuracy, in DenseNet121 we got 86.8\% accuracy, in ResNet50 we
got 94.7\% accuracy, in VGG-19 we got 93.3\% accuracy and lastly in VGG-16 we got 88.6\%
accuracy. As the results showed, after 50 epochs, RestNet50 got the highest score among the
other models with a validation accuracy of 94.7\%. Afterwards we compared all models' accuracy
and loss graph together, where we can see that VGG-19 and ResNet50 were the Good-Fit than
the other models. So for our purpose we are proposing to use the LIME for getting rid of the very
lessened percentage in accuracy. Thereupon, in this research authors mentioned and showed how
they have used Deep Neural Network Leveraging Explainable Artificial Intelligence to reduce
the amount of Glaucoma Patients through early detection. Since there still is no known approach
to prevent glaucoma, glaucoma-related blindness or major visual loss can be avoided if the
condition is detected at an early stage. As now the AI has been improved and gained reliability in
the medical sector so as per research it can be prevented by early detection. To sum up, we can
say this research has achieved the goal to bring more accuracy, reliability and committed to
improving more in Glaucoma diagnosis to make a difference in human life and contribute
accordingly.